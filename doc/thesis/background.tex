\chapter{Background and Related Work}

Most of this section will be completed in parallel with the implementation part.
This will help to basically introduce (only) the concepts required for understanding the work done.

Discuss how functional programming and structuring implementation into separate building blocks helps protocol design.
Also, benefits of type safety should not degrade performance. TODO: this though, should probably be part of hypothesis, somehow, so as to properly use it in eval section.

Explain the \textit{libOS} techniques from Mirage and how they can be pulled in from the OS world into crafting protocol implementations.
\begin{itemize}
  \item what is Mirage and how it is useful -- libraries and abstractions
  \item how this would support the expectations of the thesis
  \item features/library descriptions I need to introduce
\end{itemize}
Link functional programming language explanations and Mirage to choice of OCaml as language for project.

Discuss constraints of project in context of MPhil -- why choose only one protocol (Nigori) and the limits of compiling code to various platforms (OCaml for native, JS for browser).
Highlight ultimate choice of instantiation of protocol as Nigori.
Link to the RFC specification and how it was used -- probably talk about modifications brought to it later.
Touch upon the actual timeline of the Nigori project and versions throughout: Python, Java, now DART client.
Express how, using OCaml and Mirage libraries from a design perspective can lead to fixed server implementations and shared code that can aid clients across different platforms -- examples: native OCaml one, browser version in JS.

\begin{itemize}
  \item details of why Nigori is a good project, both as protocol and idea
  \item RFC and state(s) of project
  \item explain how it all works and the individual pieces
  \begin{itemize}
    \item client and server
    \item used crypto
    \item communication requirements
    \item messages and serialization
  \end{itemize}
\end{itemize}
