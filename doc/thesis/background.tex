\chapter{Background and Related Work}
The key insight for achieving good protocol design is understanding the two most crucial themes that should underpin the development process: \textit{simplicity} and \textit{modularity} \cite{ProtocolDesign}.

Simplicity refers to the idea that a well design protocol will often be made up of smaller pieces. In order to properly understand the working of the protocol, at large, it should suffice to understand the individual pieces, in isolation, together with the respective interactions between them. Moreover, these pieces should be specialized; they should have one clear functionality, which should potentially be easy to verify and guarantee.

Modularity, on the other hand, is supposed to complement simplicity. The concept implies that, in building a complex feature, it should require multiple smaller individual pieces. However, these should function as black boxes, assuming a well specified input and output: perfectly fine in isolation, to the extent that they can be both developed, maintained or replaced, accordingly, without affecting the integrity of the whole.

\section{Fast Prototyping}
TODO: Considering the above, we've chosen to use functional programming throughout this project. Details about modularity, composition, type safety, verifyability.

TODO: Talk about development funnel, many potentially formally verified components at the bottom (encryption?), narrow waist functional core implementation, broad spectrum of applications or target platforms through separate compilation. Potentially touch upon IP funnel.

TODO: Iterative design, Spiral Model of software development. Experiment with a \wip protocol. Decide to work on an implementation from scratch, while specification is still under development, as this would explicitly give the required context.

\section{Choice of Protocol}
TODO: Choice of protocol for Nigori. Details about initial draft \cite{NigoriDraft} and the \wip RFC \cite{NigoriRFC}. Timeline and constant changes / versions motivating the iterative design outlook. Project motivation fits the requirement for flexibility and deploying to multiple platforms. Describe internal and how it works. Describe explicit pieces to be well understood.
\begin{itemize}
  \item client and server
  \item used crypto (describe important pieces to understand in advance, cite RFCs)
  \item communication requirements (REST interface, HTTP requests and responses)
  \item messages, encoding and serialization (Base64 encoding, JSON / protobuf)
\end{itemize}

In light of the objectives for this thesis, we decided to analyze the Nigori protocol.
The project was initially started at Google, with a draft specification by Ben Laurie in 2010 \cite{NigoriDraft}.
It proposed a protocol (and an implicit system revolving around it) that would allow users to login to a server and use it for data storage, yet using encryption such that the server would be unaware of what the data actually represented.
The timeline of the project later involved an initial proof--of--concept implementation in python by Alastair Beredsford (TODO: need to cite timeline and potential location?).
This was followed by a complete implementation in Java (TOOD: need to give location and authors?), in parallel with the development of an RFC for the protocol specification \cite{NigoriRFC}.
Furthermore, in parallel with this thesis, a DART (TODO: cite DART) based version of the required client for the protocol was being developed.

As a protocol, Nigori was envisioned out of the increasingly prevalent use of the cloud as a direct storage platform and the direct impact that has on the privacy of both users and their respective data.
One the one hand, in such an environment, the cloud is one centralized entity for storing application specific data -- including potentially sensitive user data.
Nevertheless, the actualy physical representation of the storage can be completely decentralized, to the extent of data lying in separate geographic regions.
This raises questions of the security and integrity of data, if the respective servers get compromised.

One the other hand, in parallel with the adoption of using cloud storage, the number of diferent types of clients accessing the respective data has also featured an increase in diversity.
As such, we now have different native mobile OS applications, various traditional (PC/laptop) operating systems and their native applications, as well as various browser counterparts, all serving as clients, potentially sending data back into the cloud.
In this context, being able to develop the core of a protocol once and have it compiled and used consistently and with strong guarantees across platforms, becomes a very strong incentive.
What is more, Nigori was created assuming the diversity of clients, both generically, and particularly, with each individual user potentially having different devices, running the same applications, that would require both proper syncrhonization mechanisms and strong privacy of data.

Thus, given both the volatility of the project development, coupled with the multi--platform requirements, Nigori was a perfect candidate for an implementation based off of a continuous development specification.

By using Nigori, applications must be aware of two major components: the local client and the remote server.
The client is primarily responsible for properly encrypting the user data and ensuring communication with the server is done approprietely.
What is more, the client also makes use of distributed version control primitives (TODO: git, papers?) in order to guarantee that content is properly syncronized across a user's different devices and potential conflicts are able to be properly mitigated.
As such, the protocol becomes highly client--centric, placing the responsibility of both security and conflict resolution on the user end.
Nevertheless, this is meant to increase the overall safety of the user data, as, removing control from the server, implicitly places control in the hands of the user (client).

Since security is the primary responsibility of the client, most of the encryption requirements are placed on the application clients.
As discussed in the specifications, the protocol requires a couple of primitives in order to function appropriately.
\begin{description}
  \item[SHA1]
  \item[HMAC]
  \item[PBKDF]
  \item[DSA]
\end{description}

TODO: discuss client side crypto (single password requirement, server mitigations)

TODO: introduce key derivations after password recovery techniques; this is the primary mechanism to guarantee the identity of the user

TODO: talk about the four types of keys being derived: kuser, kenc, kmac, kiv

TODO: talk about data encryption (key, revision, data)

%%%%%%%%%%%%%%%%%%%%%%%%%%%
TODO: The server, on the other hand, is solely responsible for properly authenticating connecting users and handling their requests.

TODO: discuss REST and rest interface; communication is primarily done via a RESTful interface

TODO: discuss messages; separate REST endpoints cover different functionality; each piece of functionality comes with its respective message

TODO: discuss encoding and serialization; the transmitted messages must be properly formated and understood by both server and client; Base64 encoding for content; JSON / protobuf for serialization

\section{Further Methodology}
TODO: Bring in Mirage and libOS techniques. Explain how Mirage works. Describe needed library parts. Talk about how, using it, the core becomes stable and enforcing modularity. Make choice of OCaml concrete, as Mirage libraries are directly available.

TODO: Support functional programming further, as choice for implementation. Bring support for OCaml and describe needed pieces for reader. Moreover, using supporting tools around the language, we can compile to different platforms from same core code -- native OCaml, JS for browser.
