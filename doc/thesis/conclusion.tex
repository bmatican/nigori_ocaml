\chapter{Conclusion and Future Work} \label{chapter:conclusion}
As with our initial plans, a strong motivator in performing the work in this thesis was working with protocols, systems and tools which well still under development, to help asses their feasibility and ease of use.
In this regard, working with the Mirage libraries and respective tools surrounding the project was quite interesting and offered a good deal of insights.
Most of the direct libraries involved in the project proved to be sufficiently mature for turning a protocol specification into a working, albeit proof--of--concept, implementation.
Several of the more peripheral tools and libraries left aspects to be desired, however, that was somewhat to be expected, considering OCaml is not as much of a mainstream programming language, and, implictly, that will be reflected on the support and third party available software.

Work on the Nigori protocol, per se, however, was much to expectations.
As such, we were able to quickly iterate through the implementation of the protocol, until a working prototype became a reality.
Moreover, due to the way in which the underlying pieces were crafted, the code could then be properly reused and abstracted so as to allow for a Javascript based client to make use of the same structure and functionality required of the core of the protocol.
Albeit, the cross-compilation step to achieve this did require some tweaking efforts, the fact that we were able to primarily write one single implementation, with abstracted pieces of functionality and then target it to a completely different platform, was yet again, a proof--of--concept that this is achievable.

Nevertheless, the advantages of leveraging Mirage leave plenty of room for future work.
The most important avenue to pursue being, of course, using Mirage for what it was primarily built for: unikernel delopment over a hypervisor.
As such, Nigori can be bundled up, together with all its required pieces of functionality, into an application kernel and then deployed, as is, on cloud infrastructures.
This would have the advantage of potentially greatly increasing server performance.

Finally, as previously mentioned, the client was implemented in a restricted fashion, without support for the syncronization requirements that Nigori is meant to support.
As such, for a full-fledged version of the protocol specification that side of the functionality needs to be added as well.
