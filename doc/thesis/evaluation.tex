\chapter{Evaluation} \label{chapter:evaluation}
This chapter is split into two sections.
On the one hand, we discuss the theoretical outcomes of the thesis.
These are both actual modifications brought back onto the protocol specification in light of the work done throughout this thesis, as well as direct feedback on the tools and systems in use, such as Mirage -- as a whole and its set of libraries.
On the other hand, we also show the results of a practical analysis of the performance of the obtained implementation.

\section{Theoretical}
TODO: discuss what was contributed to the tools, systems and RFC.

TODO: Iterative process: discuss about modifications brought back to the RFC for various reasons (clarity, insufficient information, erroneous pieces); show improvements brought forward toward the further development of the RFC (SSH like handshake for deciding encryption mechanics, fixed set of DSA parameters...)

\section{Practical}
TODO: discuss setting for tests

TODO: discuss choice of what to test

\subsection{Storage}
TODO: discuss extra settings, discuss choices of keys and such

TODO: talk about results, interpret, hypothesis, expectations...

\subsection{Latency}
TODO: setting

TODO: each individual test alone

TODO: talk about results, interpret, hypothesis, expectations...
